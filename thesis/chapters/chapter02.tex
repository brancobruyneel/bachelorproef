\chapter{Research}

\section{Wat is Rust?}

Rust is een gecompileerde multi-paradigma programmeertaal bedacht door Graydon Hoare en is begonnen
als een project van Mozilla Research. Geïnspireerd door de programmeertalen C en C++, maar kent
weinig syntactische en semantische gelijkenissen tegenover deze talen. Rust focust zich met name op
snelheid, veiligheid, betrouwbaarheid en productiviteit. Dit wordt gerealiseerd door gebruik te
maken van een krachtig typesysteem en een borrow checker. Hiermee kan Rust een hoog niveau van
geheugenveiligheid garanderen zonder een garbage collector nodig te hebben. Rust beoogt moderne
computersystemen efficiënter te benutten. Hiervoor maakt het onder meer gebruik van het ownership
systeem dat geheugen in een blok toewijst en daarnaast strikt toeziet op de stacktoewijzing.
Hierdoor kunnen problemen zoals stackoverflows, bufferoverflows en niet-geïnitialiseerd geheugen
voorkomen worden. Verder staat Rust ook geen null-pointers, dangling-pointers of data-races toe in
veilige code.

\subsection{Syntax} 

Voor velen die zich niet herkennen in programmeertalen zoals C++, Haskell of OCaml lijkt Rust een
aparte syntax te hebben in tegenstelling tot conventionele talen. Laat ons even kijken naar een paar
syntactische voorbeelden in Rust.


Hier een simpel voorbeeld dat \mintinline{rust}{"Hello world!"} schrijft naar de standaard output.

\begin{listing}[h]
\begin{minted}{rust}
fn main() {
  println!("Hello, world!");
}
\end{minted}
\caption{Hello, world!}
\end{listing}

\clearpage

Merk op dat \mintinline{rust}{println!} geen functie is maar een macro. Geïnspireerd door de
functionele programmeertaal Scheme, zijn macro’s een manier van code schrijven dat andere code
schrijft, wat bekend staat als metaprogramming. Metaprogramming is handig voor het verminderen van
code dat u zelf hoeft te schrijven en onderhouden, wat ook een van de rollen is van functies. Toch
verschillen macro’s met functies. Zo kunnen macro’s een variabel aantal parameters hebben en worden
ze uitgebreid vooraleer de compiler de betekenis van de code interpreteert.

\begin{listing}[h]
\begin{minted}{rust}
#[derive(Debug)]
struct Rectangle {
  width: u32,
  height: u32,
}

impl Rectangle {
  fn square(size: u32) -> Rectangle {
    Rectangle: {
      width: size,
      height: size,
    }
  }
  fn area(&self) -> u32 {
    self.width * self.height
  }
}

fn main() {
  let rect1 = Rectangle::square(4);
  println!(
    "The area of the rectangle is {} square pixels.",
    rect1.area()
  );
}

\end{minted}
\caption{structs}
\end{listing}

Een \mintinline{rust}{struct} in Rust is gelijkaardig als een Object in object georiënteerde
programmeertalen. Het wordt gebruikt om samenhorende waarden te groeperen en optioneel kan men
associated functions implementeren. Associated functions die geen self als hun eerste parameter
hebben zijn geen methodes en kunnen gebruikt worden als constructors die een nieuwe instantie
retourneren van de struct.

Rust heeft geen null pointers tenzij men een null pointer wil dereferentieren (dan moet die in een
\mintinline{rust}{unsafe} blok worden geplaatst). Als alternatief voor null maakt Rust gebruik van
een \mintinline{rust}{Option} type waarmee gekeken kan worden of een pointer wel
\mintinline{rust}{Some} of geen \mintinline{rust}{None} waarde bevat. Dit
kan afgehandeld worden door syntactische sugar, zoals het if let statement om toegang te krijgen tot
het innerlijke type, in dit geval een string:

\begin{listing}[h]
\begin{minted}{rust}
fn main() {
  let name: Option<String> = None;
  // If name was not None, it would print here.
  if let Some(name) = name {
    println!("{}", name);
  }
}
\end{minted}
\caption{\mintinline{rust}{Option} type}
\end{listing}

Naast de \mintinline{rust}{if} en \mintinline{rust}{else} controlestructuren is er ook match en
\mintinline{rust}{if} \mintinline{rust}{let}. match is vergelijkbaar met een
\mintinline{javascript}{switch} statement uit andere talen. Het neemt een waarde en test het tegen
een serie van patronen. Op basis van welk patroon er overeenkomt wordt de code uitgevoerd. Patronen
kunnen opgemaakt worden uit waarden, variabel namen, wildcards, en veel meer. De power van match
komt van het feit dat de compiler zal bevestigen bij het compileren dat alle mogelijke gevallen zijn
afgehandeld. Soms wil je niet alle gevallen expliciet afhandelen en wil je slechts een patroon
afhandelen terwijl je de rest negeert. In dat geval kan je \mintinline{rust}{if let} gebruiken, wat
minder boilerplate code is dan match.

\begin{listing}[h]
\begin{minted}{rust}
let message = match maybe_digit {
  Some(x) if x < 10 => process_digit(x),
  Some(x) => process_other(x),
  None => panic!(),
};

let config_max = Some(3u8);
if let Some(max) = config_max {
  println!("The maximum is configured to be {}", max);
}
\end{minted}
\caption{\mintinline{rust}{if let} \& \mintinline{rust}{match} operators}
\end{listing}

\clearpage

\subsection{Geheugen}

Bij vele programmeertalen hoef je geen zorgen te maken over het geheugen gebruik. Dit is mogelijk
door een garbage collector te gebruiken die voortdurend zoekt naar niet langer gebruikt geheugen
terwijl het programma loopt. In andere talen, moet de programmeur het geheugen expliciet toewijzen
en vrijmaken. Rust gebruikt geen van beide methodes en komt met een uniek concept genaamd ownership.
Hiermee kan het geheugen veiligheid garanderen zonder een garbage collector nodig te hebben. 

Vooraleer we verder gaan is het belangrijk dat we de twee begrippen genaamd stack en heap begrijpen.
De twee datastructuren maken deel uit van het geheugen en zijn beschikbaar voor uw code om te
gebruiken tijdens runtime, maar ze zijn op verschillende manieren gestructureerd. Wat maakt dat de
ene zorgt voor snellere dataopslag dan de andere. 

De stack slaat waarden op in de volgorde waarin hij ze krijgt en verwijdert de waarden in de
omgekeerde volgorde. Dit wordt aangeduid als last in, first out. Alle gegevens die op de stack
worden opgeslagen moeten een bekende, vaste grootte hebben. Gegevens waarvan de grootte op het
moment van compileren onbekend is of die van grootte kunnen veranderen, moeten in plaats daarvan op
de heap worden opgeslagen. 

De heap is minder georganiseerd: als je gegevens op de heap zet, vraag je een bepaalde hoeveelheid
ruimte aan. De memory allocator vindt een lege plek in de heap die groot genoeg is, markeert die als
in gebruik, en geeft een pointer terug, dat is het adres van die locatie. Dit proces wordt
“allocating on the heap” genoemd en wordt soms afgekort als gewoon allocating. Het "pushen" van
waarden op de stack wordt niet beschouwd als allocating. Omdat de pointer naar de heap een bekende,
vaste grootte heeft, kun je de pointer op de stack opslaan, maar als je de eigenlijke gegevens wilt
hebben, moet je de pointer volgen. 

Het efficiëntste is dus om data naar de stack weg te schrijven dan naar de heap, omdat de allocator
nooit hoeft te zoeken naar een plaats om nieuwe gegevens op te slaan. Die plaats is altijd bovenaan
de stack. Het alloceren van ruimte op de heap vergt meer werk, omdat de allocator eerst een ruimte
moet vinden die groot genoeg is om de gegevens op te slaan en dan de boekhouding moet doen om de
volgende allocatie voor te bereiden. 

Rust heeft dus een voorkeur om zijn variabelen weg te schrijven naar de stack. Maar zoals gezegd kan
je niet elke variabele wegschrijven naar de stack, daarom is er een onderscheid tussen simpele en
complexe types. Bij simpele types ken je de grootte voor het compileren, daarmee worden ze dan ook
opgeslagen op de stack. In tegenstelling tot complexe types waarbij de grootte kan veranderen,
worden ze opgeslagen in de heap.

De simpele types zijn:
\begin{itemize}
  \item Integer
  \item Floating-point
  \item Boolean
  \item Character
  \item Tuple
  \item Array (ze hebben dus een vaste grootte in Rust)
\end{itemize}

Nu de begrippen zijn opgeklaard kunnen we kijken naar het ownership systeem. Het systeem bestaat uit
drie regels: 
\begin{enumerate}
  \item Elke waarde in Rust heeft een variabele die de owner wordt genoemd 
  \item Er kan maar een owner per keer zijn 
  \item Wanneer de owner buiten scope gaat, zal de waarde worden verwijderd 
\end{enumerate}

Deze regels worden gecontroleerd bij het compileren. Als een van de regels wordt overtreden zal het
programma niet compileren.  

Laat ons eens kijken naar een simpel voorbeeld.

\begin{listing}[h]
\begin{minted}{rust}
{
  let s = String::from("hello"); // s is geldig vanaf deze lijn
  // doe iets met s
} // hier eindigt de scope, en s is niet langer geldig
\end{minted}
\caption{ownership}
\end{listing}

Als we de regels volgen, wordt de variabele s owner over de string literal \mintinline{rust}{hello}.
De variabele \mintinline{rust}{s} blijft geldig zolang hij binnen de scope wordt aangeroepen. Op het
einde van de scope zal Rust de \mintinline{rust}{drop} functie uitvoeren en dus het geheugen terug
vrijgeven. Dit is de basis van hoe het ownership systeem werkt in Rust. Dit was een zeer simpel
voorbeeld en in de realiteit komen er natuurlijk nog wat eigenaardigheden bij kijken.

\subsection{Ecosysteem}

\blindtext

\section{Wat is WebAssembly \& Hoe werkt het?}

Om interactieve webapps te creëren gebruik je tegenwoordig javascript. Ondanks de succesvolle
inspanningen van browsermakers om hun javascript-engines in elke versie weer wat efficiënter te
maken, is dat voor veel toepassingen nog niet genoeg. Google dan kwam in 2011 met Native Client
(NaCI), een sandbox voor het efficiënt en veilig uitvoeren van gecompileerde C en C++ code in de
browser, onafhankelijk van het besturingssysteem van de gebruiker. Het bracht prestaties en
low-level controle van native code naar moderne webbrowsers, zonder de veiligheid en portabiliteit
van het web op te offeren. [1] 

Mozilla wilde de platformonafhankelijkheid van javascript echter niet verlaten, en begon daarom in
2013 aan een andere aanpak: asm.js, een subset van javascript die browsers heel efficiënt kunnen
uitvoeren. Je compileert dan een webapp uit een taal zoals C naar asm.js, en je browser voert dit
dan als gewone javascript uit. 

Het voordeel van asm.js is dat het gewoon al in alle webbrowsers werkte, maar Mozilla botste tegen
de snelheidsgrenzen van javascript aan. Omdat javascript een tekstformaat heeft, vraagt het parsen
veel rekenkracht, zeker op mobiele toestellen met een wat zwakkere processor. En zo werd in 2015
WebAssembly (afgekort Wasm) geboren, een binair instructieformaat voor een stack-gebaseerde virtuele
machine in je webbrowser. Het is ontworpen als een overdraagbaar compilatiedoel voor
programmeertalen, waardoor het gebruik op het web mogelijk wordt voor client- en servertoepassingen.
[2] [3]

\subsection{Doel}

Het is dus geen nieuwe programmeertaal, maar een binair formaat voor uitvoerbare programma’s. Het
wordt gecreëerd als een open standaard binnen de W3C WebAssembly Community Group met de volgende
doelstellingen: 
\begin{itemize}
  \item Snel, efficiënt en overdraagbaar - wasm code kan op bijna-native snelheid worden uitgevoerd
    op verschillende platforms door gebruik te maken van gemeenschappelijke hardware mogelijkheden. 

  \item Leesbaar en foutopspoorbaar - wasm is een lage assembleertaal, maar het heeft een menselijk
    leesbaar tekstformaat (aan de specificatie wordt nog gewerkt) waarmee code met de hand kan
    worden geschreven, bekeken en foutopsporing mogelijk is. 

  \item Veilig - wasm is gespecificeerd om te worden uitgevoerd in een veilige, sandboxed omgeving.
    Net als andere webcode, zal het de browser's same-origin en permissies beleid afdwingen. 

  \item Maak het web niet kapot - wasm is zo ontworpen dat het goed samengaat met andere
    webtechnologieën en achterwaartse compatibiliteit behoudt.
\end{itemize}

\subsection{Hoe werkt het?}

Nu we weten wat wasm op een hoog niveau inhoudt, is het ook goed om eens praktisch te kijken hoe het
precies werkt. Er zijn namelijk een aantal opties voor het compileren naar wasm: 

\begin{itemize}
  \item C/C++ applicatie omzetten naar wasm met Emscripten 

  \item wasm rechtstreeks op assembly niveau schrijven of genereren 

  \item een Rust applicatie schrijven en wasm als compilatie target gebruiken 

  \item AssemblyScript gebruiken, wat vergelijkbaar is met Typscript en compileert naar een wasm
    binary 
\end{itemize}

In deze bachelorproef werd het technisch onderzoek uitgevoerd in Rust. Met zijn kleine runtime,
betrouwbaar en rijk typesysteem is het een van de populairste keuzes voor het bouwen van webapps met
WebAssembly. Dus laat ons even kijken naar een voorbeeld hoe we vanuit Rust javascript functies
kunnen gebruiken en andersom.

Een van de moeilijkste onderdelen van het werken met WebAssembly is om verschillende soorten waarden
in en uit functies te krijgen. Dat komt omdat WebAssembly momenteel slechts twee types kent:
integers en floating point getallen. 

Dit betekent dat je niet zomaar een string in een WebAssembly functie kunt stoppen. In plaats
daarvan moet je een aantal stappen doorlopen om een string voor te stellen als getallen. Als je
complexere types hebt, zul je zelfs een ingewikkelder proces hebben om de gegevens heen en weer te
sturen. Gelukkig bestaat er de library wasm-bindgen die deze stappen voor ons doet. Met een paar
annotaties aan je Rust code, zal het automatisch de code maken die nodig is (aan beide kanten) om
complexere types te laten werken.

\section{Welke front- \& backend frameworks zijn er ter beschikking?}
Om efficiënt webapplicaties te bouwen heb je natuurlijk een degelijk framework nodig die voor jou al
het zware werk doet. Gelukkig heeft Rust mits zijn jonge jaren, al een aardig aantal frameworks ter
beschikking voor het bouwen van webapplicaties.  

Dit zijn de top 3 front- en backend frameworks. Gerangschikt naar mate van hun populariteit (GitHub
stars).

\subsection{Frontend}

\subsubsection{Yew - 21k} 

Yew is het populairste frontend framework met over 21k GitHub stars. Het beschikt
over een component-gebaseerd framework dat het makkelijk maakt om interactieve UI’s te maken.
Ontwikkelaars die ervaring hebben met frameworks als React en Elm zouden zich helemaal thuis moeten
voelen bij het gebruik van Yew. Naast de aangename developer experience brengt het ook geweldige
prestaties met zich mee. Dit bereiken ze door het minimaliseren van het aanroepen naar de DOM en
door ontwikkelaars te helpen om gemakkelijk taken te offloaden naar achtergrond threads met behulp
van web workers. 

De documentatie over de huidige release versie 19.0 is uitgebreid geschreven.  Er kan zelfs al
gekeken worden naar de volgende release “Next”, die refereert naar de master branch. In de Next
versie is belangrijkste nieuwe feature SSR, want de huidige versie heeft alleen maar CSR.

\clearpage

\subsubsection{Dixous - 3.8k}

Dixous is een opkomend en jong UI framework als concurrent voor Yew. Het is net als Yew ontworpen om
React-achtig te zijn. Zo heeft het ook een component-gebaseerde architectuur, state management,
props en nog veel meer. Buiten de verschillende syntax tegen over Yew, heeft Dixous nog een aantal
troeven zoals:  
\begin{itemize}
  \item de keuze tussen JSX-achtige of hun eigen macro gebaseerde RSX syntax als templating systeem
    ingebouwde globale state en error handler 

  \item components en hooks kunnen worden hergebruikt voor te renderen op het web, desktop, mobiel,
    server en meer 

  \item uitgebreide inline documentatie 

  \item SSR 
\end{itemize}

\subsubsection{Seed - 3.3k}

Seed is een frontend framework voor het maken van prestatiegerichte en betrouwbare webapps die een
Elm-achtige architectuur heeft. Het heeft een minimale configuratie en boilerplate, en heeft
duidelijke documentatie die het voor iedereen gemakkelijk maakt om mee te beginnen.  

Ook Seed gebruikt een eigen templating systeem met een macro syntax waardoor Rustaceans zich meteen
thuis voelen. Dit betekent dat linting, formatteren en commentaar geven zullen werken, en het is
allemaal in Rust. Dit in tegenstelling tot een JSX-achtige syntax (zoals die van Yew) die
afhankelijk is van IDE-extensies om de developer experience te verbeteren. 

Seed beschikt niet over SSR en heeft nog geen plannen om het te implementeren. 

\subsection{Backend}

\subsubsection{Rocket – 17.2k}

Rocket is een populair webframework dat het voor developers gemakkelijk maakt om snelle webapps te
schrijven zonder te bezuinigen op veiligheid, flexibiliteit of functionaliteiten. Het heeft
ondersteuning voor het testen van libraries, cookies, streams, routes, templates, databases, ORMs,
boilerplates, en nog veel meer. Rocket heeft ook een grote en actieve community. 

\subsubsection{Actix-web – 14.1k}

Net als Rocket, is Actix een ander krachtig backend web framework. Actix heeft een
architectuurpatroon gebaseerd op het actor-systeem van Rust en is goed uitgerust voor het bouwen van
schrijfdiensten en micro apps. Het heeft ondersteuning voor routing, middleware, testen, WebSockets,
automatisch server reloading en kan gehost worden op NGINX. Actix kan worden gebruikt om een
volledige web app en API te bouwen. 

\clearpage

\subsubsection{Axum – 4.7k}

Alhoewel er nog populairdere frameworks zijn dan Axum verdient het zeker een plaats in de top 3.
Axum is deel van het populaire Tokio project, met sponsors als aws, azure en facebook. Tokio is een
asynchrone runtime voor Rust, die de bouwstenen biedt die nodig zijn voor het schrijven van
netwerktoepassingen. 

Axum is een laag bovenop Tokio’s HTTP client genaamd hyper, die een relatief low-level library is en
bedoeld is al bouwsteen voor libraries en applicaties. Het framework focust op ergonomie en
modulariteit. Wat het nog onderscheidt met andere frameworks, is dat het geen eigen middleware
systeem heeft maar gebruikt in plaats daarvan de tower::Service module van het Tokio project. Dit
betekent dat Axum gratis timeouts, tracing, compressie, authorisatie en meer krijgt. 

Dit alles maakt met zijn jonge 1-jarige leeftijd toch een library om naar uit te kijken. 

\section{Hoe bouw je een web app in Rust?}

Sinds Yew het populairste framework is en ook is gebruikt als framework voor het bouwen van de speed
typing applicatie, zal de vraag "Hoe bouw je een Webapplicatie in Rust" beantwoord worden met Yew
als framework. Het bouwen van een webapplicatie in een ander framework zal in grote lijnen hetzelfde
zijn. Dit zal een praktische kijk zijn op hoe we Yew kunnen gebruiken voor het bouwen van
Webapplicaties.  

In dit voorbeeld gaan we een simpele Todo applicatie bouwen.

\subsection{Tools installeren}

\subsubsection{Rust}
Om Rust te installeren hebben we de rustup toolchain installer nodig. Met het onderstaand script kan
je het installeren op jouw UNIX machine. Als je Rust al hebt staan maak dan zeker dat je de laatste
versie hebt door rustup update uit te voeren.

\begin{minted}{bash}
curl --proto '=https' --tlsv1.2 -sSf https://sh.rustup.rs | sh
\end{minted}

\subsubsection{WebAssembly}

Rust kan source codes compileren voor verschillende "targets" (m.a.w verschillende processors). Het
compilatie target voor een browser gebaseerd WebAssembly heet wasm32-unkown-unkown. Het volgende
commando zal het WebAssembly target toevoegen aan je development environment.

\begin{minted}{bash}
rustup target add wasm32-unknown-unknown
\end{minted}

\clearpage

\subsubsection{Trunk}

De documentatie van Yew raad aan om Trunk te gebruiken voor het beheren van deployment en packaging.

\begin{minted}{bash}
cargo install --locked trunk
\end{minted}

Nu ons development enviroment opgezet is, kunnen we een nieuw cargo project aanmaken.

\begin{minted}{bash}
cargo new yew-app
cd yew-app
\end{minted}

Om te verifieren dat het Rust environment juist is opgezet, voer je het initiele project uit met de
cargo build tool. Na de output van de het build process, zou je normaal "Hello, world!" te zien
krijgen.

\begin{minted}{bash}
cargo run
\end{minted}

\subsubsection{Statische pagina}

Om deze simpele command line applicatie naar een basis Yew web applicatie te convertern, zijn er een
paar aanpassingen nodig. Pas de volgende bestanden aan als volgt:

\begin{listing}[h]
\begin{minted}{toml}
[package]
name = "yew-app"
version = "0.1.0"
edition = "2021"

[dependencies]
yew = "0.19"
\end{minted}
\caption{Cargo.toml}
\end{listing}

\clearpage

\begin{listing}[h]
\begin{minted}{rust}
use yew::prelude::*;

#[function_component(App)]
fn app() -> Html {
    html! {
        <h1>{ "Hello World" }</h1>
    }
}

fn main() {
    yew::start_app::<App>();
}
\end{minted}
\caption{main.rs}
\end{listing}

Maak nu een index.html aan in de root folder van het project.

\begin{listing}[h]
\begin{minted}{html}
<!DOCTYPE html>
<html lang="en">
  <head> </head>
  <body></body>
</html>
\end{minted}
\caption{index.html}
\end{listing}

\subsubsection{Start de development server}

Voer het volgende commando uit om de applicatie te builden en lokaal te draaien.

\begin{minted}{bash}
trunk serve --open
\end{minted}

\subsection{Statische pagina}

\subsubsection{Bouwen van HTML}

Yew maakt gebruik van de procedurele macro's van Rust en biedt ons een syntax die lijkt op JSX (een
uitbreiding van JavaScript waarmee u HTML-achtige code kunt schrijven in JavaScript) om de opmaak te
maken.

\clearpage

\subsubsection{Converteren van HTML naar Rust}

Aangezien we al een vrij goed idee hebben van hoe onze website eruit zal zien, kunnen we onze
mentale opzet eenvoudig vertalen naar een voorstelling die compatibel is met html!. Als je
eenvoudige HTML kunt schrijven, moet het geen probleem zijn om markeringen in html! te schrijven.
Het is belangrijk op te merken dat de macro op een paar punten verschilt van HTML:
\begin{itemize}
  \item Uitdrukkingen moeten tussen accolades staan (\{ \})
  \item Er mag maar één root node zijn. Als je
    meerdere elementen wilt hebben zonder ze in een container te wikkelen, wordt een lege
    tag/fragment (<> ... </>) gebruikt
  \item Elementen moeten goed worden afgesloten.
\end{itemize}

We willen een layout bouwen die er ongeveer zo uitziet in ruwe HTML:

\begin{minted}{html}
<main>
  <h1>My todo list</h1>
  <ul>
    <li>
      <input type="checkbox" />
      <label>Take dog out for a walk</label>
    </li>
    <input type="checkbox" />
    <label>Feed the cats</label>
    <li>
      <input type="checkbox" />
      <label>Take out the trash</label>
    </li>
    <li>
      <input type="checkbox" />
      <label>Water plants</label>
    </li>
  </ul>
</main>
\end{minted}

\clearpage

Laten we nu deze HTML in html! omzetten. Type (of kopieer/plak) het volgende knipsel in de body van
de app functie, zodat de waarde van html! wordt geretourneerd door de functie.

\begin{listing}[h]
\begin{minted}{rust}
#[function_component]
pub fn App() -> Html {
  html! {
    <main>
      <h1>{ "My todo list" }</h1>
      <ul>
        <li>
          <input type="checkbox"/>
          <label> { "Take dog out for a walk" } </label>
        </li>
          <input type="checkbox"/>
          <label> { "Feed the cats" } </label>
        <li>
          <input type="checkbox"/>
          <label> { "Take out the trash" } </label>
        </li>
        <li>
          <input type="checkbox"/>
          <label> { "Water plants" } </label>
        </li>
      </ul>
    </main>
  }
}
\end{minted}
\caption{app.rs}
\end{listing}

\clearpage

\subsubsection{Components}

Components zijn de bouwstenen van Yew applicaties. Door components te combineren, die weer uit
andere components kunnen worden opgebouwd, bouwen we onze applicatie. Door onze components te
structureren voor herbruikbaarheid en ze generiek te houden, kunnen we ze in meerdere delen van onze
applicatie gebruiken zonder code of logica te hoeven dupliceren.

In feite is de app functie die we tot nu toe hebben gebruikt een component, genaamd App. Het is een
"function component". Er zijn twee verschillende soorten componenten in Yew:
\begin{itemize}
  \item Struct Components 
  \item Function Components
\end{itemize}

In deze tutorial zullen we function components gebruiken.

Laten we nu onze App component opsplitsen in kleinere componenten. We kunnen onze Todo lijst
opspliten in 2 components genaamd Task en TaskList.

\begin{listing}[h]
\begin{minted}{rust}
#[derive(Properties, Debug, PartialEq)]
pub struct TaskProps {
  pub id: String,
  pub title: String,
  pub completed: bool,
}

#[function_component]
pub fn Task(
  TaskProps {
    id,
    title,
    completed,
  }: &TaskProps,
) -> Html {
  html! {
    <li>
      <input
        type="checkbox"
        id={id.clone()}
        checked={*completed}
      />
      <label
        for={id.clone()}>{title.clone()}
      </label>
    </li>
  }
}
\end{minted}
\caption{task.rs}
\end{listing}

\clearpage

Let op de parameters van onze Task function component. Een function component heeft slechts één
argument dat zijn "props" (kort voor "properties") definieert. Props worden gebruikt om gegevens
door te geven van een ouder component naar een kind component. In dit geval is TaskProps een struct
die de props definieert.

\begin{listing}[h]
\begin{minted}{rust}
#[derive(Properties, PartialEq)]
pub struct TaskListProps {
  pub children: Children,
}

#[function_component]
pub fn TaskList(TaskListProps { children }: &TaskListProps) -> Html {
  html! {
    <ul>
      { for children.iter() }
    </ul>
  }
}
\end{minted}
\caption{task\_list.rs}
\end{listing}

Nu kunnen we onze App component updaten met onze nieuwe components Task \& TaskList.

\begin{listing}[h]
\begin{minted}{rust}
#[function_component]
pub fn App() -> Html {
    let tasks = vec![
        html! { 
          <Task 
            id={"1"}
            title={"Take dog out for a walk"}
            completed={true}
          /> 
        },
        html! { 
          <Task 
            id={"2"}
            title={"Feed the cats"}
            completed={false}
          /> 
        },
        html! { 
          <Task 
            id={"3"}
            title={"Water the plants"}
            completed={false}
          /> 
        },
    ];
    html! {
        <main>
            <h1>{ "My todo list" }</h1>
            <TaskList>
                {tasks}
            </TaskList>
        </main>
    }
}
\end{minted}
\caption{app.rs}
\end{listing}

\clearpage

\subsubsection{Interactief}
Momenteel doet onze applicatie niet veel anders dan onze voor gedefineerde lijst te tonen.
Uiteindlijk willen we onze taken uit de todo lijst kunnen schrappen. Om deze interactie te laten
werken zullen we een aantal zaken aanpassen.

Om bij te houden of de gebruiker de taak geschrapt heeft al dan niet, zullen we een
\mintinline{rust}{completed_state} gebruiken met de \mintinline{rust}{use_state} hook. Zo verliezen
we niet de waarde van de variabele \mintinline{rust}{completed_state} mocht het component
re-renderen.

\begin{minted}{rust}
#[function_component]
pub fn Task(
    ...
) -> Html {
    let completed_state = use_state(|| *completed);
    ...
}
\end{minted}

Het volgende is natuurlijk het \mintinline{rust}{onclick} event afhandelen als de gebruiker op het
label of input element klikt. Hiervoor hebben we een \mintinline{rust}{Callback} functie nodig die
onze \mintinline{rust}{completed_state} aanpast naargelang de vorige state. Voor we de
\mintinline{rust}{completed_state} kunnen gebruiken in onze Callback closure functie moeten we die
eerst clonen. De reden hiervoor is het keyword \mintinline{rust}{move} die voor de closure
parameters staat, \mintinline{rust}{move} zorgt
ervoor dat alle references die gebruikt worden in de closure scope hun waarden worden verplaatst
binnen de scope. Dus om te voorkomen dat we onze \mintinline{rust}{completed_state} nergens meer
kunenn gebruiken clonen we eerst de state.

Daarnaast kunnen we ook een css class meegeven met de \mintinline{rust}{classes!} macro van yew, die
conditioneel het html label zal doorstrepen al dan niet.

\clearpage

\begin{listing}
\begin{minted}{rust}
#[function_component]
pub fn Task(
    TaskProps {
        id,
        title,
        completed,
    }: &TaskProps,
) -> Html {
    let completed_state = use_state(|| *completed);

    let onclick = {
        let completed_state = completed_state.clone();

        Callback::from(move |_| {
            if *completed_state {
                completed_state.set(false);
            } else {
                completed_state.set(true);
            }
        })
    };

    let completed_class = {
        if *completed_state {
            "line-through"
        } else {
            ""
        }
    };

    html! {
        <li>
            <input
                {onclick}
                type="checkbox"
                id={id.clone()}
                checked={*completed_state}
            />
            <label
                class={classes!(completed_class)}
                for={id.clone()}>{title.clone()}
            </label>
        </li>
    }
}
\end{minted}
\caption{task.rs}
\end{listing}

\clearpage

\subsubsection{Data extern ophalen}

In de speed typing applicatie komen de code snippets van een API in plaats van hard gecodeerd te
zijn in de frontend. Laten we onze todo lijst ophalen van een externe bron. Hiervoor moeten we de
volgende crates toevoegen:
\begin{itemize}
  \item reqwasm voor het maken van de fetch call.
  \item serde met derive functies Voor het de-serialiseren van het JSON antwoord
  \item wasm-bindgen-futures voor het uitvoeren van Rust Future als een Promise
\end{itemize}

Laten we de dependencies in het Cargo.toml bestand bijwerken:

\begin{minted}{toml}
[dependencies]
yew = { git = "https://github.com/yewstack/yew/", features = ["csr"] }
serde = { version = "1.0", features = ["derive"] }
wasm-bindgen-futures = "0.4"
\end{minted}

Pas de \mintinline{rust}{TaskProps} struct aan om de \mintinline{rust}{Deserialize} trait af te leiden.

\begin{minted}{rust}
#[derive(Properties, Debug, PartialEq, Deserialize)]
pub struct TaskProps {
  pub id: String,
  pub title: String,
  pub completed: bool,
}
\end{minted}

Als laatste stap moeten we onze App component updaten om de fetch request te maken in plaats van
hardcoded data te gebruiken.

Hier gebruiken we de \mintinline{rust}{use_effect_with_deps} hook met lege dependencies als tweede
parameter om de tasks slechts eenmaal op te halen bij de eerste render van het App component. In de
closure gebruiken we \mintinline{rust}{wasm-bindgen-futures} om de Javascript Promise die de
\mintinline{rust}{Request::get} genereert om te zetten naar een Future type. Met de
\mintinline{rust}{Request::get} halen we de JSON todos op en slaan ze op als een
\mintinline{rust}{vec} met \mintinline{rust}{TaskProps}. Vervolgens vullen we onze
\mintinline{rust}{tasks} state met de \mintinline{rust}{fetched_tasks} en zullen de tasks worden
weergeven in de browser!

\begin{listing}
\begin{minted}{rust}
#[function_component]
pub fn App() -> Html {
let tasks = use_state(std::vec::Vec::new);
  {
    let tasks = tasks.clone();
    use_effect_with_deps(move |_| {
      wasm_bindgen_futures::spawn_local(async move {
        let fetched_tasks: Vec<TaskProps> = Request::get("url")
          .send()
          .await
          .unwrap()
          .json()
          .await
          .unwrap();

        tasks.set(fetched_tasks.iter().take(10).map(|props| {
            html! { 
              <Task 
                id={props.id.clone()} 
                title={props.title.clone()} 
                completed={props.completed} 
              /> 
            }
          }).collect()
        );
      });
      || ()
    }, ());
  }

  html! {
      <main>
          <h1>{ "My todo list" }</h1>
          <TaskList>
              { (*tasks).clone() }
          </TaskList>
      </main>
  }
}
\end{minted}
\caption{app.rs}
\end{listing}

\clearpage

\section{Hoe bouw je een API in Rust?}

Volgend op de Todo applicatie die we gebouwd hebben in "Hoe bouw je een web app in Rust?",
zullen we een REST API bouwen die de taken uit een database zal halen en opslaan. Als API framework
zullen we Actix-web gebruiken, samen met diesel.rs als ORM voor het beheren van de database. Om het
simpel te houden gebruiken we net zoals in de speed typing test applicatie SQLite als database.

\clearpage

\section{Is Rust productie klaar?}

Wanneer een taal “productie klaar” benoemd kan worden is eigenlijk een moeilijke vraag om te
beantwoorden. Wat “productie klaar” betekent hangt af van verschillende factoren voor een specifiek
domein. Rust schittert bijvoorbeeld in productief veilige en performante applicaties te bouwen. Top
bedrijven zoals Amazon, Google, Microsoft kunnen dit beamen sinds ze ondertussen al een aantal jaar
Rust gebruiken. Zo heeft Amazon in 2018 Firecracker gelanceerd. Een open source
virtualisatietechnologie die AWS Lambda en andere serverloze aanbiedingen aandrijft. Daarnaast
gebruiken ze Rust ook voor services te leveren zoals Amazon Simple Storage Service (Amazon S3),
Amazon Elastic Compute Cloud (Amazon EC2), Amazon CloudFront, en meer. Amazon heeft dan ook samen
met Google, Microsoft, Huawei en Mozilla de Rust foundation opgericht in 2020, met een missie om het
developen van Rust te ondersteunen. We kunnen dus concluderen dat de taal op zich wel een stabiele
omgeving is. Dit onderzoek gaat natuurlijk wel over hoe we Rust kunnen gebruiken voor het bouwen van
webapplicaties. We richten onze focus dan ook op dit domein. We zullen de onderzoeksvraag “Is Rust
productie klaar?” opsplitsen in twee delen namelijk frontend en backend. 

\subsection{Backend }

Laten we beginnen met de backend. Bij het bouwen van hedendaagse webapplicaties heb je over het
algemeen 2 zaken nodig, een API en een database. Uiteraard is dit niet altijd het geval maar laten
we onze focus hierop leggen. 

Om een web API te bouwen gebruik je gewoonlijk een web framework. Het huidige rust ecosysteem heeft
een aantal stabiele web frameworks ter beschikking. Uit de onderzoeksvraag “Hoe bouw je een API in
Rust?” hebben we gekozen voor Actix-web te gebruiken als framework. Actix-web voorziet alles wat je
kan verwachten van een web framework, van routing en middleware, tot templating en JSON/form
afhandeling. Het is ook een van de snelste web frameworks ter beschikking. Zo staat actix momenteel
op de 5de plaats bij een benchmark ranking van alle populaire web frameworks (ook andere
programmeertalen dan Rust). 

Het actix framework ondersteunt verschillende soorten databases zoals mongodb, postgres, redis,
sqlite, graphql, elasticsearch en meer. De ene database heeft meer ondersteuning dan de andere, maar
naarmate de tijd vordert zal de ondersteuning alleen maar verbeteren. 

\subsection{Frontend}

Bij de frontend komt er wat meer bij kijken. Er zijn namelijk twee opties waarvoor Rust gebruikt kan
worden: 
\begin{itemize}
  \item Een gedeelde in Rust schrijven en compileren naar wasm om vervolgens te importeren als een node
  \item module in een bestaande Javascript applicatie. 
  \item Een volledige frontend in Rust schrijven met een web framework als Yew en compileren naar wasm 
\end{itemize}

\clearpage

De eerste optie is meteen ook de optie waarvoor wasm het meest zal gebruikt worden. Zo kan je
Javascript blijven gebruiken voor de UI en wasm voor de logica van de applicatie. Om mogelijks meer
performatie te winnen en correctere code te schrijven. Deze optie wordt al volop in productie
gebruikt door meerdere bedrijven: 
\begin{itemize}
  \item 1Password – gebruikt wasm om hun browser plugin te versnellen 
  \item Figma – gebruikt wasm om hun originele C++ applicatie te exporteren naar de browser 
  \item SketchUp – gebruikt wasm om een 3d modelling tool op het web te draaien, wat snelle en
    voorspelbare prestaties verreist 
\end{itemize}

De andere optie, een volledige frontend bouwen met Rust, is een optie die momenteel meer gebruikt
wordt voor hobby projecten dan voor productie. Een van de redenen hiervoor is dat het ecosysteem nog
zeer jong is. Het is bijvoorbeeld niet te vergelijken met het ecosysteem van populaire JS
frameworks, wat ook logisch is. Toch kan een bedrijf er een voordeel uithalen. Bijvoorbeeld als hun
volledige stack geschreven is in Rust, kunnen ze delen van hun codebase zoals database entiteiten
hergebruiken in de frontend. Nog een voordeel is dat beide teams maar een taal hoeven te kennen en
zo elkaar kunnen helpen bij veelvoorkomende fouten.  

Het heeft natuurlijk ook zijn nadelen. De compileer tijd kan soms op lopen tot enkele minuten. 
