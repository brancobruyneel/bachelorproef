\chapter{Research}

\section{Wat is Rust?}

Rust is een gecompileerde multi-paradigma programmeertaal bedacht door Graydon Hoare en is begonnen
als een project van Mozilla Research. Geïnspireerd door de programmeertalen C en C++, maar kent
weinig syntactische en semantische gelijkenissen tegenover deze talen. Rust focust zich met name op
snelheid, veiligheid, betrouwbaarheid en productiviteit. Dit wordt gerealiseerd door gebruik te
maken van een krachtig typesysteem en een borrow checker. Hiermee kan Rust een hoog niveau van
geheugenveiligheid garanderen zonder een garbage collector nodig te hebben. Rust beoogt moderne
computersystemen efficiënter te benutten. Hiervoor maakt het onder meer gebruik van geheugenbeheer
dat geheugen in een blok toewijst en daarnaast strikt toeziet op de stacktoewijzing. Hierdoor kunnen
problemen zoals stackoverflows, bufferoverflows en niet-geïnitialiseerd geheugen voorkomen worden.
Verder staat Rust ook geen null-pointers, dangling-pointers of data-races toe in veilige code.

\subsection{Syntax} 

Voor velen die zich niet herkennen in programmeertalen zoals C++, Haskell of OCaml lijkt Rust een
aparte syntax te hebben in tegenstelling tot conventionele talen. Laat ons even kijken naar een paar
syntactische voorbeelden in Rust.


Hier een simpel voorbeeld dat \mintinline{rust}{"Hello world!"} schrijft naar de standaard output.

\begin{listing}[h]
  \begin{minted}{rust}
    fn main() {
      println!("Hello, world!");
    }
  \end{minted}
  \caption{Hello, world!}
\end{listing}

\clearpage

Merk op dat \mintinline{rust}{println!} geen functie is maar een macro. Geïnspireerd door de
functionele programmeertaal Scheme, zijn macro’s een manier van code schrijven dat andere code
schrijft, wat bekend staat als metaprogramming. Metaprogramming is handig voor het verminderen van
code dat u zelf hoeft te schrijven en onderhouden, wat ook een van de rollen is van functies. Toch
verschillen macro’s met functies. Zo kunnen macro’s een variabel aantal parameters hebben en worden
ze uitgebreid vooraleer de compiler de betekenis van de code interpreteert.

\begin{listing}[h]
  \begin{minted}{rust}
    #[derive(Debug)]
    struct Rectangle {
      width: u32,
      height: u32,
    }

    impl Rectangle {
      fn square(size: u32) -> Rectangle {
        Rectangle: {
          width: size,
          height: size,
        }
      }
      fn area(&self) -> u32 {
        self.width * self.height
      }
    }

    fn main() {
      let rect1 = Rectangle::square(4);
      println!(
        "The area of the rectangle is {} square pixels.",
        rect1.area()
      );
    }

  \end{minted}
  \caption{structs}
\end{listing}

Een \mintinline{rust}{struct} in Rust is gelijkaardig als een Object in object georiënteerde
programmeertalen. Het wordt gebruikt om samenhorende waarden te groeperen en optioneel kan men
associated functions implementeren. Associated functions die geen self als hun eerste parameter
hebben zijn geen methodes en kunnen gebruikt worden als constructors die een nieuwe instantie
retourneren van de struct.

Rust heeft geen null pointers tenzij men een null pointer wil dereferentieren (dan moet die in een
\mintinline{rust}{unsafe} blok worden geplaatst). Als alternatief voor null maakt Rust gebruik van
een \mintinline{rust}{Option} type waarmee gekeken kan worden of een pointer wel
\mintinline{rust}{Some} of geen \mintinline{rust}{None} waarde bevat. Dit
kan afgehandeld worden door syntactische sugar, zoals het if let statement om toegang te krijgen tot
het innerlijke type, in dit geval een string:

\begin{listing}[h]
  \begin{minted}{rust}
    fn main() {
      let name: Option<String> = None;
      // If name was not None, it would print here.
      if let Some(name) = name {
        println!("{}", name);
      }
    }
  \end{minted}
  \caption{\mintinline{rust}{Option} type}
\end{listing}

Naast de \mintinline{rust}{if} en \mintinline{rust}{else} controlestructuren is er ook match en
\mintinline{rust}{if} \mintinline{rust}{let}. match is vergelijkbaar met een
\mintinline{javascript}{switch} statement uit andere talen. Het neemt een waarde en test het tegen
een serie van patronen. Op basis van welk patroon er overeenkomt wordt de code uitgevoerd. Patronen
kunnen opgemaakt worden uit waarden, variabel namen, wildcards, en veel meer. De power van match
komt van het feit dat de compiler zal bevestigen bij het compileren dat alle mogelijke gevallen zijn
afgehandeld. Soms wil je niet alle gevallen expliciet afhandelen en wil je slechts een patroon
afhandelen terwijl je de rest negeert. In dat geval kan je \mintinline{rust}{if let} gebruiken, wat
minder boilerplate code is dan match.

\begin{listing}[h]
  \begin{minted}{rust}
    let message = match maybe_digit {
      Some(x) if x < 10 => process_digit(x),
      Some(x) => process_other(x),
      None => panic!(),
    };

    let config_max = Some(3u8);
    if let Some(max) = config_max {
      println!("The maximum is configured to be {}", max);
    }
  \end{minted}
  \caption{\mintinline{rust}{if let} \& \mintinline{rust}{match} operators}
\end{listing}

\clearpage

\subsection{Geheugen}

Bij vele programmeertalen hoef je geen zorgen te maken over het geheugen gebruik. Dit is mogelijk
door een garbage collector te gebruiken die voortdurend zoekt naar niet langer gebruikt geheugen
terwijl het programma loopt. In andere talen, moet de programmeur het geheugen expliciet toewijzen
en vrijmaken. Rust gebruikt geen van beide methodes en komt met een uniek concept genaamd ownership.
Hiermee kan het geheugen veiligheid garanderen zonder een garbage collector nodig te hebben. 

Vooraleer we verder gaan is het belangrijk dat we de twee begrippen genaamd stack en heap begrijpen.
De twee datastructuren maken deel uit van het geheugen en zijn beschikbaar voor uw code om te
gebruiken tijdens runtime, maar ze zijn op verschillende manieren gestructureerd. Wat maakt dat de
ene zorgt voor snellere dataopslag dan de andere. 

De stack slaat waarden op in de volgorde waarin hij ze krijgt en verwijdert de waarden in de
omgekeerde volgorde. Dit wordt aangeduid als last in, first out. Alle gegevens die op de stack
worden opgeslagen moeten een bekende, vaste grootte hebben. Gegevens waarvan de grootte op het
moment van compileren onbekend is of die van grootte kunnen veranderen, moeten in plaats daarvan op
de heap worden opgeslagen. 

De heap is minder georganiseerd: als je gegevens op de heap zet, vraag je een bepaalde hoeveelheid
ruimte aan. De memory allocator vindt een lege plek in de heap die groot genoeg is, markeert die als
in gebruik, en geeft een pointer terug, dat is het adres van die locatie. Dit proces wordt
“allocating on the heap” genoemd en wordt soms afgekort als gewoon allocating. Het "pushen" van
waarden op de stack wordt niet beschouwd als allocating. Omdat de pointer naar de heap een bekende,
vaste grootte heeft, kun je de pointer op de stack opslaan, maar als je de eigenlijke gegevens wilt
hebben, moet je de pointer volgen. 

Het efficiëntste is dus om data naar de stack weg te schrijven dan naar de heap, omdat de allocator
nooit hoeft te zoeken naar een plaats om nieuwe gegevens op te slaan. Die plaats is altijd bovenaan
de stack. Het alloceren van ruimte op de heap vergt meer werk, omdat de allocator eerst een ruimte
moet vinden die groot genoeg is om de gegevens op te slaan en dan de boekhouding moet doen om de
volgende allocatie voor te bereiden. 

Rust heeft dus een voorkeur om zijn variabelen weg te schrijven naar de stack. Maar zoals gezegd kan
je niet elke variabele wegschrijven naar de stack, daarom is er een onderscheid tussen simpele en
complexe types. Bij simpele types ken je de grootte voor het compileren, daarmee worden ze dan ook
opgeslagen op de stack. In tegenstelling tot complexe types waarbij de grootte kan veranderen,
worden ze opgeslagen in de heap.

De simpele types zijn:
\begin{itemize}
  \item Integer
  \item Floating-point
  \item Boolean
  \item Character
  \item Tuple
  \item Array (ze hebben dus een vaste grootte in Rust)
\end{itemize}

Nu de begrippen zijn opgeklaard kunnen we kijken naar het ownership systeem. Het systeem bestaat uit
drie regels: 
\begin{enumerate}
  \item Elke waarde in Rust heeft een variabele die de owner wordt genoemd 
  \item Er kan maar een owner per keer zijn 
  \item Wanneer de owner buiten scope gaat, zal de waarde worden verwijderd 
\end{enumerate}

Deze regels worden gecontroleerd bij het compileren. Als een van de regels wordt overtreden zal het
programma niet compileren.  

Laat ons eens kijken naar een simpel voorbeeld.

\begin{listing}[h]
  \begin{minted}{rust}
    {
      let s = String::from("hello"); // s is geldig vanaf deze lijn
      // doe iets met s
    } // hier eindigt de scope, en s is niet langer geldig
  \end{minted}
  \caption{ownership}
\end{listing}

Als we de regels volgen, wordt de variabele s owner over de string literal \mintinline{rust}{hello}.
De variabele \mintinline{rust}{s} blijft geldig zolang hij binnen de scope wordt aangeroepen. Op het
einde van de scope zal Rust de \mintinline{rust}{drop} functie uitvoeren en dus het geheugen terug
vrijgeven. Dit is de basis van hoe het ownership systeem werkt in Rust. Dit was een zeer simpel
voorbeeld en in de realiteit komen er natuurlijk nog wat eigenaardigheden bij kijken.

\subsection{Ecosysteem}

\blindtext

\section{Wat is WebAssembly \& Hoe werkt het?}

Om interactieve webapps te creëren gebruik je tegenwoordig javascript. Ondanks de succesvolle
inspanningen van browsermakers om hun javascript-engines in elke versie weer wat efficiënter te
maken, is dat voor veel toepassingen nog niet genoeg. Google dan kwam in 2011 met Native Client
(NaCI), een sandbox voor het efficiënt en veilig uitvoeren van gecompileerde C en C++ code in de
browser, onafhankelijk van het besturingssysteem van de gebruiker. Het bracht prestaties en
low-level controle van native code naar moderne webbrowsers, zonder de veiligheid en portabiliteit
van het web op te offeren. [1] 

Mozilla wilde de platformonafhankelijkheid van javascript echter niet verlaten, en begon daarom in
2013 aan een andere aanpak: asm.js, een subset van javascript die browsers heel efficiënt kunnen
uitvoeren. Je compileert dan een webapp uit een taal zoals C naar asm.js, en je browser voert dit
dan als gewone javascript uit. 

Het voordeel van asm.js is dat het gewoon al in alle webbrowsers werkte, maar Mozilla botste tegen
de snelheidsgrenzen van javascript aan. Omdat javascript een tekstformaat heeft, vraagt het parsen
veel rekenkracht, zeker op mobiele toestellen met een wat zwakkere processor. En zo werd in 2015
WebAssembly (afgekort Wasm) geboren, een binair instructieformaat voor een stack-gebaseerde virtuele
machine in je webbrowser. Het is ontworpen als een overdraagbaar compilatiedoel voor
programmeertalen, waardoor het gebruik op het web mogelijk wordt voor client- en servertoepassingen.
[2] [3]

\subsection{Doel}

Het is dus geen nieuwe programmeertaal, maar een binair formaat voor uitvoerbare programma’s. Het
wordt gecreëerd als een open standaard binnen de W3C WebAssembly Community Group met de volgende
doelstellingen: 
\begin{itemize}
  \item Snel, efficiënt en overdraagbaar - wasm code kan op bijna-native snelheid worden uitgevoerd
    op verschillende platforms door gebruik te maken van gemeenschappelijke hardware mogelijkheden. 

  \item Leesbaar en foutopspoorbaar - wasm is een lage assembleertaal, maar het heeft een menselijk
    leesbaar tekstformaat (aan de specificatie wordt nog gewerkt) waarmee code met de hand kan
    worden geschreven, bekeken en foutopsporing mogelijk is. 

  \item Veilig - wasm is gespecificeerd om te worden uitgevoerd in een veilige, sandboxed omgeving.
    Net als andere webcode, zal het de browser's same-origin en permissies beleid afdwingen. 

  \item Maak het web niet kapot - wasm is zo ontworpen dat het goed samengaat met andere
    webtechnologieën en achterwaartse compatibiliteit behoudt.
\end{itemize}

\subsection{Hoe werkt het?}

Nu we weten wat wasm op een hoog niveau inhoudt, is het ook goed om eens praktisch te kijken hoe het
precies werkt. Er zijn namelijk een aantal opties voor het compileren naar wasm: 

\begin{itemize}
  \item C/C++ applicatie omzetten naar wasm met Emscripten 

  \item wasm rechtstreeks op assembly niveau schrijven of genereren 

  \item een Rust applicatie schrijven en wasm als compilatie target gebruiken 

  \item AssemblyScript gebruiken, wat vergelijkbaar is met Typscript en compileert naar een wasm
    binary 
\end{itemize}

In deze bachelorproef werd het technisch onderzoek uitgevoerd in Rust. Met zijn kleine runtime,
betrouwbaar en rijk typesysteem is het een van de populairste keuzes voor het bouwen van webapps met
WebAssembly. Dus laat ons even kijken naar een voorbeeld hoe we vanuit Rust javascript functies
kunnen gebruiken en andersom.

Een van de moeilijkste onderdelen van het werken met WebAssembly is om verschillende soorten waarden
in en uit functies te krijgen. Dat komt omdat WebAssembly momenteel slechts twee types kent:
integers en floating point getallen. 

Dit betekent dat je niet zomaar een string in een WebAssembly functie kunt stoppen. In plaats
daarvan moet je een aantal stappen doorlopen om een string voor te stellen als getallen. Als je
complexere types hebt, zul je zelfs een ingewikkelder proces hebben om de gegevens heen en weer te
sturen. Gelukkig bestaat er de library wasm-bindgen die deze stappen voor ons doet. Met een paar
annotaties aan je Rust code, zal het automatisch de code maken die nodig is (aan beide kanten) om
complexere types te laten werken.

\section{Welke front- \& backend frameworks zijn er ter beschikking?}
\section{Hoe bouw je een web frontend in Rust?}
\section{Hoe bouw je een API in Rust?}
\section{Is Rust productie klaar?}
