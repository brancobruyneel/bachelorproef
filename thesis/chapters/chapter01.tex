\chapter{Inleiding}

Al 6 jaar op een rij is Rust verkozen tot meest geliefde programmeertaal bij de developers enquête
van Stack Overflow. Daarbij verdienen Rust developers ook nog eens de op drie na hoogste salaris en
versloegen daarmee Python en Typescript. \cite{so_enquete} De taal produceert razendsnelle en
veilige code zonder een runtime of garbage collector te gebruiken. De veilige code bereikt Rust door
het ownership model waardoor u vele soorten bugs kunt elimineren tijdens het compileren.

Naast de prestaties legt het ook de nadruk op productiviteit. De taal komt met geweldige
documentatie, een eerste klasse compiler met nuttige foutmeldingen en top-notch tooling. Zo heeft
het een ingebouwde package manager en build tool, slimme multi-editor ondersteuning met
auto-completion en type inspections, een auto-formatter en meer. \cite{rustlang}

De twee laatste domeinen genaamd networking en WebAssembly komen aanbod in deze bachelorproef. 
Er wordt namelijk onderzocht naar de huidige status van Rust voor het bouwen van webapplicaties en
doet daarvoor onderzoek naar de volgende deelvragen:
\begin{itemize}
  \item Wat is Rust?
  \item Wat is WebAssembly \& Hoe werkt het?
  \item Welke front- \& backend frameworks zijn er ter beschikking?
  \item Hoe bouw je een webapp in Rust?
  \item Hoe bouw je een API in Rust?
  \item Is Rust productie klaar?
\end{itemize}

Met de kennis uit het onderzoek wordt er een technisch onderzoek uitgevoerd door een demo applicatie
te bouwen. De demo is een simpele 'speed typing' webapp waar bij de gebruiker een random code snippet
zo snel mogelijk moet over typen. Verder wordt er gereflecteerd op het behaalde resultaat met
externen uit de praktijk. Uit de reflectie wordt een advies geformuleerd en een conclusie die de
onderzoeksvraag beantwoordt.

De methode die in deze studie is gehanteerd, is een gemende aanpak gebaseerd op onderzoek uit
verschillende bronnen, praktische resultaten en informatie verkregen uit het werkveld.
