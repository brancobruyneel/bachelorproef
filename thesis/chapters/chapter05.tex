\chapter{Advies}

\section{Bruikbaarheid}

Een van de grote doelen in 2018 was voor de Rust community om een webtaal te worden. Door zich te
richten op WebAssembly, kan Rust nu ook worden uitgevoerd op het web net als JavaScript. Dit
betekent niet dat Rust Javascript volledig zal vervangen. Javascript biedt nog altijd meer voordelen
bij het bouwen van een web app. Dit is omdat JS een goede keuze is voor de meeste dingen. Het is
snel en gemakkelijk om aan de slag te gaan met JavaScript. Bovendien is er een levendig ecosysteem
vol met JavaScript ontwikkelaars die ongelooflijk innovatieve benaderingen hebben gecreëerd voor
verschillende problemen op het web.

Nu in 2022 zijn er al een aantal tools die het makkelijk maken om Rust te gebruiken op het web. We
hebben \mintinline{rust}{webpack} en \mintinline{rust}{trunk} die voor ons de Rust code bundelt en
omzet naar wasm. Daarnaast kunnen we eenvoudig met javascript praten met behulp van
\mintinline{rust}{wasm-bindgen} en \mintinline{rust}{web-sys}. Toch is het nog verre van perfect.
Uiteindelijk willen we de browser API direct kunnen aanspreken vanuit WebAssembly. Een voorgestelde
oplossing waar nu aan wordt gewerkt is interface types \cite{wasm_interfaces}. Het probleem dat het
tracht op te lossen is het vertalen van waarden tussen verschillende types wanneer een module met
een andere module praat (of rechtstreeks met een host, zoals de browser). Want de huidige versie van
WebAssembly kan alleen maar met nummers praten. 

Wanneer WebAssembly dus niet met Javascript of de browser moet praten kan het dus profiteren van de
uitzonderlijk snelle prestaties van een low level taal als Rust.

Buiten de browser kan Rust gewoon worden gebruikt in (micro) services voor het web waar prestaties
van cruciaal belang zijn. Het zal wat extra werk vereisen om 'correcte' en geheugen veilige code te
schrijven volgens de borrow checker, maar dit zal resulteren in extra prestaties en een robuust
systeem.

\clearpage

\section{Aanbevelingen}

In de praktijk moeten er nog een aantal stappen ondernomen worden voor 

Toch zullen bedrijven er waarschijnlijk meer baat bij hebben om delen van hun huidige applicatie te
vervangen door wasm. Zo kunnen ze 



de bruikbaarheid en toepasbaarheid van je vooropgestelde oplossingen.

De basis is er om het 


is het bruikbaar ja? ja
toepasbaarheid voor kleine  

welke concrete aanbevelingen het werkveld volgens jou kan ondernemen op basis van jouw onderzoeksresultaten.

welk stappenplan het werkveld hierbij zou kunnen gebruiken.

welke tools je voor het werkveld ontwikkeld hebt.

andere relevante adviezen voor het werkveld, gebaseerd op je onderzoek.


stappenplan

Mocht ik het onderzoek opnieuw starten zou ik een aantal zaken anders aangepakt hebben.

De eerste paar weken van het onderzoek had ik besteed aan het lezen van het Rust boek. Terwijl er
genoeg kleine voorbeelden waren van code die je kon meevolgen in het boek, is dit niet genoeg om
een goed inzicht te krijgen in de taal. Daarom raad ik aan om tijdens het lezen van het boek per
hoofdstuk zelf met een paar kleine demo's te komen.

Wat ook niet onbelangrijk is, zoals vermeld in de reflectie, is het schrijven van testen. Mocht ik
de tijd genomen hebben om eerst de testen te schrijven en dan mijn functies. Zou ik hoogst
waarschijnlijk een pak tijd gewonnen hebben. De testen zou ik hierdan geschreven hebben alleen voor
de api en parser tool, sinds dat Yew naar mijn weten nog geen ondersteuning heeft voor integratie
testen.

Vooraleer ik het project begon dacht ik ook dat dit een 'simpele' applicatie ging worden, maar heb
uiteindelijk te veel tijd besteed aan de logica en werking van het spel. Mocht ik een andere soort
webapp gemaakt hebben kon ik meer tijd besteden aan authenticatie, ci/cd, wasm optimaliseren, en zo
verder.

schrijf testen


In sectie \ref{frameworks} sprak ik over dat Axum een interessant framework was om naar uit te
kijken. Met dat Tokio zo'n populair async framework is en Axum deel is van het Tokio project, zou ik
eens de api herschrijven in Axum. Om dan te kijken of het voor- of nadelen biedt tegenover actix.



Het onderzoek uit deze bachelorproef toont aan wat er mogelijk is met Rust voor het bouwen van web
applicaties.

Het eerste advies dat ik al kan geven is begin niet aan het bouwen van WebApplicaties zonder eerst
de taal onder te knie te hebben
bouw een bijvoorbeeld eerst een CLI applicatie
