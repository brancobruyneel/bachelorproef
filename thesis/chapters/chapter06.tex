\chapter{Conclusie}

Met wat extra werk tegenover andere webprogrammeertalen, biedt Rust snelle prestaties, een rijk
typesysteem en correcte code. De prestaties verschillen echter per domein. Indien het gebruikt wordt
voor web services te schrijven, dan kan er geprofiteerd worden van de uitstekende prestaties samen
met een kleine 'runtime'. Met die reden gebruiken top bedrijven Rust in (micro) services.

Met die zelfde redenen wordt Rust gezien als de taal om te compileren naar WebAssembly. Zoals gezien
zijn er twee opties om wasm in de browser te draaien. Bij de eerste optie schrijven we een gedeelte
van de applicatie in wasm, die gewoonlijk krachtige prestaties verwacht zonder de browser aan te
spreken. Hier benutten we de volledige kracht van wasm samen met een low level taal als Rust
die veel efficiënter is dan Javascript. Deze optie zien we dan ook het vaakst terug in productie.

De tweede optie, een volledige webapp schrijven in Rust, verliest prestaties. Sinds wasm nog niet
direct de browser API kan aanspreken en dus hiervoor Javascript gebruikt. Toch hebben we gezien dat
een framework als Yew, betere benchmark resultaten haalt tegenover populaire Javascript frameworks.
Terwijl Julius en Cecicle beiden Yew gebruiken in productie, blijft het over het algemeen bij hobby
projecten door het jonge framework en ecosysteem er rondt.

We kunnen dus concluderen dat de huidige status van Rust voor het bouwen van webapplicaties nog
volop in ontwikkeling is.
