\chapter{Reflectie}
\label{reflectie}

Het doel van de module 'Research Project' was om onderzoek te doen naar de vraag "Wat is de huidige
status van Rust voor het bouwen van webapplicaties?". In dit hoofdstuk reflecteer ik op het
resultaat van het onderzoek en vergelijk de bevindingen uit de praktijk. Daarvoor heb ik contact
opgenomen met twee core maintainers van het gebruikte web framework genaamd Yew. Zij hebben beide
ook professionele ervaring met het bouwen van webapplicaties in Rust.

De eerste core maintainer is Julius Lungys, hij is een Rust developer bij het Fintech bedrijf
Nikulipe. Al hun software is geschreven met Rust, de software is voornamelijk backend gericht maar
voor de admin websites gebruiken ze Yew. Julius was ook te gast bij \citetitle{podcast}
\cite{podcast} met een episode over Yew, waar ik interessante inzichten gekregen heb over dit
onderzoek.

Cecile Tonglet is de tweede core maintainer, in het dagelijkse leven ook een Rust developer
bij HMI Hydronics. Ze maken hydraulische ventielen, inclusief een "slimme" ventiel met wat embedded
code. Cecile werkt niet aan de niet aan de embedded software maar werkt aan een web app die draait 
op het slimme ventiel.

Naast de core maintainers heb ik ook gereflecteerd met Emiel Van Severen, een mastersstudent
Industrieel Ingenieur in de Informatica. Hij heeft nog geen professionele ervaring met Rust maar
is zich zeer bekend in het ecosysteem en heeft een aantal hobby projecten in Rust.

Ook was er tijdens het onderzoek ook aardig wat feedback van de Yew community. Yew heeft een
relatief kleine maar zeer actieve community. Zo kreeg ik in hun discord instant feedback bij
problemen of vragen.

\clearpage

\section{Wat zijn de sterke en zwakke punten van het resultaat uit jouw researchproject?}

\subsection{Sterke}

Tijdens het interview met Julius Lungys, stelde ik vraag waarom hij zo enthousiast over Rust is. 
Hij antwoorde dat de taal alles heeft wat hij wilt, een eigen equivalent van npm (Cargo), ingebouwde
code formatter, tests, docs, geen garbage collector en macros.

Dit kon ik alleen maar beamen, Cargo is een geweldige package manager en maakt het beheren van
packages zeer eenvoudig. Cargo is ongetwijfeld een van Rust zijn sterkte punten en zal een positieve
impact hebben op de groei van het ecosysteem.

Rust is correct met een krachtig typesysteem, dit betekent dat alle edge cases in de code worden
opgevangen. Denk maar aan het unwrappen van een \mintinline{rust}{Option} of
\mintinline{rust}{Result} type. Gelukkig heeft Rust een intelligente compiler die suggesties geeft,
wanneer er iets fout loopt tijdens het compileren. Dit zorgt ervoor als de applicatie compileert, je
er zeker van kan zijn dat de meeste voorkomende bugs er uit zijn.

In deelvraag \ref{productie} \enquote{Is Rust productie klaar?} zagen we bij de benchmark resultaten dat
Rust efficiënter omgaat met het geheugen dan de normale Javascript frameworks. Wat natuurlijk ook
logisch is sinds Javascript een dynamische getypeerde taal is en gebruik maakt van een garbage
collector wat zorgt voor veel overhead. Samen met WebAssembly opent dit de deuren om allerhande
rekenkrachtige applicaties op het web te draaien die ervoor niet mogelijk waren.

\subsection{Zwakke}

Momenteel het grootste nadeel om productief te zijn in het bouwen van webapplicaties met Rust, is
het beperkte ecosysteem er rond. Hoe sneller een bedrijf een web app kan uitbrengen hoe minder
kosten ze hebben. Dat is een van de redenen waarom er bijvoorbeeld veel voor gestijlde component
libraries op de markt zijn voor Javascript. Yew voorziet al een paar copies van zo'n bekende
Javascript libraries maar die zijn allemaal nog te jong om te kunnen gebruiken in productie.

Samen met Rust is WebAssembly nog jong en volop in beweging. Zoals besproken in de deelvraag
\ref{productie} \enquote{Is Rust productie klaar?} is wasm nog niet perfect.

\textbf{TODO: eerst het wasm hoofdstuk nog afwerken}

Voor dat Julius aan zijn Rust carrière begon was hij een React developer. Wat hem nu stoort is de
hoeveelheid boilerplate code dat je nodig hebt in Rust om het zelfde te kunnen bereiken in React.
Zo hoeven we bijvoorbeeld voor simpele functies aan te spreken in Javascript libraries gebruiken als
\mintinline{rust}{gloo} die speelt als 'lijm' tussen Rust, wasm en Javascript.


\section{Is ‘het projectresultaat’ (incl. methodiek) bruikbaar in de bedrijfswereld}

cargo workspace

er zijn geen testen

gzipped content


\section{Wat zijn de mogelijke implementatiehindernissen voor een bedrijf?}

hosting / deployment

compilatie tijden



\section{Wat is de meerwaarde voor het bedrijf?}

als het bedrijf al rust gebruikt bij hun andere services


\section{Welke alternatieven/suggesties geven bedrijven en/of community?}


\section{Is er een economische meerwaarde aanwezig?}


\section{Wat zijn jouw suggesties voor een (eventueel) vervolgonderzoek?}

ci/cd deployment

