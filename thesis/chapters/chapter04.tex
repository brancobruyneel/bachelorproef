\chapter{Reflectie}

Het doel van de module "Research Project" was om onderzoek te doen naar de vraag "Wat is de huidige
status van Rust voor het bouwen van webapplicaties?". In dit hoofdstuk reflecteer ik op het
resultaat van het onderzoek en vergelijk de bevindingen uit de praktijk. Daarvoor heb ik contact
opgenomen met twee core maintainers van het gebruikte web framework genaamd Yew. Zij hebben beide
ook professionele ervaring met het bouwen van webapplicaties in Rust.

De eerste core maintainer is Julius Lungys, hij is een Rust developer bij het Fintech bedrijf
Nikulipe. Al hun software is geschreven met Rust, de software is voornamelijk backend gericht
maar voor de admin websites gebruiken ze Yew. Julius was ook te gast bij "The Rustacean Station Podcast"
met een episode over Yew, waar ik interessante inzichten gekregen heb over dit onderzoek.

Cecile Tonglet is de tweede core maintainer, in het dagelijkse leven ook een Rust developer
bij HMI Hydronics. Ze maken hydraulische ventielen, inclusief een "slimme" ventiel met wat embedded
code. Cecile werkt niet aan de niet aan de embedded software maar werkt aan een web app die draait 
op het slimme ventiel.

Naast de core maintainers heb ik ook gereflecteerd met Emiel Van Severen, een laatstejaarsstudent
Industrieel Ingeineur in de Informatica. Hij heeft nog geen professionele ervaring met Rust maar
is zich zeer bekend in het ecosysteem en heeft een aantal hobby projecten in Rust.

Ook was er tijdens het onderzoek ook aardig wat feedback van de Yew community. Yew heeft een
relatief kleine maar zeer actieve community. Zo kreeg ik in hun discord instant feedback bij
problemen of vragen.


\clearpage

\section{Wat zijn de sterke en zwakke punten van het resultaat uit jouw researchproject?}

\subsection{Sterke}

rust package management

sterk typesystem + compiler

rust verzekert dat alle edge cases worden afgehandeld

bv het unrwappen van een Result of option


performance gewijs hetzelfde


\subsection{Zwakke}

veel boiler plate om het zelfde te kunnen bereiken dan met react + typescript

ecosysteem nog niet ontwikkeld

de frameworks hebben vaak nog geen 1.0 versie


\section{Is ‘het projectresultaat’ (incl. methodiek) bruikbaar in de bedrijfswereld}

cargo workspace

er zijn geen testen

gzipped content


\section{Wat zijn de mogelijke implementatiehindernissen voor een bedrijf?}




\section{Wat is de meerwaarde voor het bedrijf?}

als het bedrijf al rust gebruikt bij hun andere services


\section{Welke alternatieven/suggesties geven bedrijven en/of community?}


\section{Is er een economische meerwaarde aanwezig?}


\section{Wat zijn jouw suggesties voor een (eventueel) vervolgonderzoek?}

ci/cd deployment

