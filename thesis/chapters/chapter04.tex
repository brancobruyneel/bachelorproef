\chapter{Reflectie}
\label{reflectie}

Het doel van de module 'Research Project' was om onderzoek te doen naar de vraag "Wat is de huidige
status van Rust voor het bouwen van webapplicaties?". In dit hoofdstuk wordt er gereflecteerd op het
resultaat van het onderzoek en vergelijken we de bevindingen uit de praktijk. Daarvoor is er contact
opgenomen met twee core maintainers van Yew die beide professionele ervaring hebben met het bouwen
van webapplicaties in Rust. 

De eerste core maintainer is Julius Lungys, hij is een Rust developer bij het Fintech bedrijf
Nikulipe. Al hun software is geschreven in Rust, de software is voornamelijk backend gericht maar
voor de admin websites gebruiken ze Yew. Julius was ook te gast bij \citetitle{podcast}
\cite{podcast} met een episode over Yew, waar er interessante inzichten zijn gekregen voor het
reflecteren over dit onderzoek.

Cecile Tonglet is de tweede core maintainer, in het dagelijkse leven is ze een Rust developer bij
HMI Hydronics. Ze maken hydraulische ventielen, inclusief een "slimme" ventiel met wat embedded
code. Cecile werkt niet aan de niet aan de embedded software maar werkt aan een web app die draait
op het slimme ventiel.

Naast de core maintainers, werd er ook gereflecteerd met Emiel Van Severen, een mastersstudent
Industrieel Ingenieur in de Informatica. Hij heeft nog geen professionele ervaring met Rust maar is
zich zeer bekend in het ecosysteem en heeft een aantal hobby projecten in Rust.

Ook was er tijdens het onderzoek ook aardig wat feedback van de Yew community. Yew heeft een
relatief kleine maar zeer actieve community. Zo werd er bijna instant feedback gegeven bij problemen
of vragen.

\clearpage

\section{Wat zijn de sterke en zwakke punten van het resultaat uit jouw researchproject?}

\subsection{Sterke}

Tijdens het interview met Julius Lungys, werd er de vraag gesteld waarom hij zo enthousiast is over
Rust. Hij antwoorde dat de taal alles heeft wat hij wilt, een eigen equivalent van npm (Cargo),
ingebouwde code formatter, tests, docs, geen garbage collector en macros.

Cargo is ongetwijfeld een van Rust zijn sterke punten. Het is een geweldige package manager en maakt
het beheren van packages zeer eenvoudig. Met Cargo kan u dus makkelijk nieuwe crates developen en
publiceren naar \url{crates.io}. Dit zal dus zeker een positieve impact hebben op de groei van het
ecosysteem.

Rust is correct met een krachtig typesysteem, dit betekent dat alle edge cases in de code worden
opgevangen. Denk maar aan het unwrappen van een \mintinline{rust}{Option} of
\mintinline{rust}{Result} type. Gelukkig heeft Rust een intelligente compiler die suggesties geeft
wanneer er iets fout loopt tijdens het compileren. Dit zorgt ervoor als de applicatie compileert, u
zeker kan zijn dat de meest voorkomende bugs er uit zijn.

In deelvraag \ref{productie} \enquote{Is Rust productie klaar?} zagen we dat de taal uitstekend
presteert bij de benchmarks zowel bij de frontend als backend. Als we de frontend resultaten
vergelijken met de populaire Javascript frameworks zien we dat Rust efficiënter omgaat met het
geheugen. Wat natuurlijk ook logisch is sinds Javascript een dynamische getypeerde taal is en
gebruik maakt van een garbage collector wat zorgt voor veel overhead. Samen met WebAssembly opent
dit de deuren om rekenkrachtige applicaties op het web te draaien die ervoor niet mogelijk waren.

\subsection{Zwakke}

Momenteel is het groottste nadeel het beperkte ecosysteem bij het bouwen van een frontend voor
webapplicaties. Hoe sneller een bedrijf een web app kan uitbrengen hoe minder kosten ze hebben. Dat
is een van de redenen waarom er bijvoorbeeld veel voorgestijlde component libraries op de markt zijn
voor Javascript. Yew voorziet al een paar copies van zo'n bekende Javascript libraries maar die zijn
allemaal nog te jong om te kunnen gebruiken in productie.

Naast het beperkte ecosysteem hebben de frontend frameworks in Rust vaak nog geen 1.0 versie bereikt
en zullen dus in de komende tijd nog zeker wat breaking changes met zich mee brengen.

Samen met Rust is WebAssembly nog jong en volop in beweging. Zoals besproken in de deelvraag
\ref{productie} \enquote{Is Rust productie klaar?} is wasm nog niet perfect. Zo hoeven we nog altijd
javascript te gebruiken om te praten met de DOM, maar dit is slechts tijdelijk.

Voor dat Julius aan zijn Rust carrière begon was hij een React developer. Wat hem nu stoort is de
hoeveelheid boilerplate code dat u nodig heeft in Rust om het zelfde te kunnen bereiken in React.
Zo hoeven we bijvoorbeeld voor simpele functies aan te spreken in Javascript libraries gebruiken als
\mintinline{rust}{gloo} een high level wrapper rond de \mintinline{rust}{wasm-bindgen} crate die
speelt als 'lijm' tussen Rust en Javascript.

\clearpage

\section{Is ‘het projectresultaat’ (incl. methodiek) bruikbaar in de bedrijfswereld?}

Kort gezegd, ja. De technische demo toonde aan welke stappen er ondernomen moesten worden voor een
simpele webapplicatie te bouwen. Deze stappen kunnen we beschouwen als een basis die een bedrijf kan
gebruiken voor een webapp te bouwen in Rust. De methodiek werd dan ook afgetoetst bij externen.
Julius en Cecile hadden beide geen opmerkingen over de structuur/methodiek van het project. 

Julius gebruikt zelf een gelijkaardige structuur in de praktijk. Op zijn werk gebruiken ze een grote
mono repo waar all hun services inclusief de web applicaties inzitten. Met als voordeel dat ze een
paar crates hebben met code die ze kunnen delen, bijvoorbeeld een crate database entiteiten.

De backend zal wel eerder gebruikt kunnen worden dan de frontend, sinds het gebruikte framework
(Yew) nog geen majore versie heeft bereikt. Wat maakt dat er waarschijnlijk nog breaking changes
zullen komen.


\section{Wat zijn de mogelijke implementatiehindernissen voor een bedrijf?}

De huidige stabel release van Yew heeft nog geen SSR. Dit betekent met de huidige CSR, er twee grote
nadelen zijn:
\begin{itemize}
   \item Gebruikers zullen niets kunnen zien totdat de hele WebAssembly bundel is gedownload en de
   eerste render is voltooid. Dit kan resulteren in een slechte gebruikerservaring als de gebruiker
   een traag netwerk gebruikt.

  \item Sommige zoekmachines ondersteunen geen dynamisch gerenderde web content en zij die dat wel
  doen plaatsen dynamische websites meestal lager in de zoekresultaten.
\end{itemize}
Het gebrek aan SSR kan dus gezien worden als een implementatiehindernis, indien het bedrijf zijn web
app wil optimaliseren.

Een andere implementatiehindernis kan zijn dat het team binnen het bedrijf geen ervaring heeft met
Rust. De taal opzich heeft al een steile leercurve, daarnaast moeten ze ook nog bedrijpen hoe
WebAssembly werkt. Gelukkig hebben mensen met React ervaring al een voorsprong als ze voor het Yew
framework kiezen, sinds het bijna een excate copie is.

\section{Wat is de meerwaarde voor het bedrijf?}

Als een bedrijf de frontend van een applicatie volledig in Rust schrijft, zal dat op dit moment niet
veel meerwaarde brengen. Buiten dat als een bedrijf een volledige stack heeft in Rust, ze het
makkelijk kunnen vinden om ook hun web applicaties te bouwen in Rust. 

Zoals besproken in sectie \ref{productie} \enquote{Is Rust productie klaar?} is de grootste use case
momenteel om Rust/WebAssembly te gebruiken waar er grote prestatie vereisten zijn of een gedeelte van
de applicatie als de logica in Rust schrijven.

Wat wel een meerwaarde is, is om micro services te schrijven in Rust die gebruikt worden voor het
web. Rust zijn binary grootte zonder te compileren naar WebAssembly is klein. Daarnaast is Rust ook
super snel, perfect voor kleine micro services in de cloud. Denk maar bijvoorbeeld aan AWS lambda
functies. Emiel weette mij te vertellen dat nu ook Cloudflare workers volledig in Rust geschreven
kunnen worden. \cite{cloudflare_workers}

\section{Welke suggesties geven bedrijven en/of community?}

Tijdens het zoeken naar externen om te reflecteren, is er ook een bericht geplaatst in de Yew
discord server. Zij stelden voor om de wasm code bij de server te comprimeren, zodat het niet de
client $\sim$380kB aan wasm code ongecomprimeerd laat downloaden over het internet. Een andere suggestie
van de community was om de \mintinline{rust}{reqwasm} crate te laten vallen, want die is ondertussen
toegevoegd aan de \mintinline{rust}{gloo} crate. Dit zou ook de wasm binary grootte kunnen
verkleien.

Emiel suggesteeerde om testen te schrijven, Rust heeft daarvoor een heel goed builtin test systeem.
Dat had zeker een meerwaarde kunnen zijn om bijvoorbeeld de parser tool uit te werken met behulp van
test driven development.

Julius stelde ook voor om de \mintinline{rust}{hooks} te gebruiken van \mintinline{rust}{Trunk}, op
die manier zou ik een hook kunnen schrijven die bij het bouwen van de applicatie ook de tailwindcss
scripts uitvoert. De huidige manier was via een npm script.

\section{Wat zijn jouw suggesties voor een (eventueel) vervolgonderzoek?}

Vooraleer het onderzoek begon, werd er gevraagd een contractplan in te vullen. Daarin is er
opsomming opgelijst van wat het technisch onderzoek resultaat minimaal zal bevatten. Helaas was er
tijdens het onderzoek niet genoeg tijd voor alles te realiseren. Dit zijn de criterea's die niet
gerealiseerd zijn:
\begin{enumerate}
  \item De applicatie is beveiligd met SSO
  \item Er is een profiel pagina aanwezig
  \item De statistische resultaten worden opgeslaan in een database
\end{enumerate}
In een vervolgonderzoek zou de applicatie dan ook worden kunnen uitgebreid met deze
functionaliteiten. Om de applicatie te beveiligen met SSO, zou ik OAUTH implementeren. Op die manier
kunnen gebruikers inloggen met bijvoorbeeld hun Github account.  Met een gebruikers account zal ik
ook een profiel pagina kunnen aanmaken en de statistische resultaten per gebruiker opslaan in de
database.

Tijdens het onderzoek heb ik geen aandacht gegeven aan ci/cd. Ondertussen heb ik al een kleine
pipeline opgezet om de frontend te deployen naar een Azure Static Web App en de api naar een Azure
API App. Dit is vlug opgezet en kan zeker ook verder onderzocht / geoptimaliseert worden.

Wat ik ook nog niet heb onderzocht is het optimaliseren van de wasm binary.\cite{wasm_size} Volgens
de community zijn er een aantal stappen die u kan ondernemen om uw binary te verkleinen. Wat
belangrijk is, want we hebben gezien bij de benchmark resultaten dat Yew een grotere bundle size
over het internet stuurt dan Javascript frameworks.

Als laatste zou ik proberen een javascript module als bijvoorbeeld \mintinline{javascript}{Chart.js}
gebruiken in het web project. Dit zou mogelijk moeten zijn door de \mintinline{rust}{wasm-bindgen}
crate te gebruiken om de functies van de JS library te mappen naar Rust. Met
\mintinline{javascript}{Chart.js} zou ik dan een grafiek kunnen tonen van de afgelopen resultaten.
