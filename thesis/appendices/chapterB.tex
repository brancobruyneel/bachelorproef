\chapter{Verslag Cyber Crime Unit}

Tijdens de module 'Research Project' kregen we de kans om twee sessies bij te wonen van
gastsprekers. Op 11 januari 2022 gaf Francis Nolf een online sessie over de belgische Cyber Crime
Unit (CCU). 

In 2001 kregen we voor het eerst in Belgie een regionale en federale CCU. De federale eenheid staat
in voor het beleid en is het centraal punt voor internetcriminaliteit, opleidingen en steun aan de
regionale CCU's. Een van hun taken is bijvoorbeeld onderzoek naar nieuwe ICT-systemen en forensische
onderzoeksprogramma's. De regionale eenheid is belast met operationeel werk en verricht bijvoorbeeld
forensisch onderzoek van pc-apparatuur. Door nood aan specialisaties zijn de teams onderverdeeld in
vier groepen:
\begin{itemize}
  \item \textbf{Open Source Intelligence (OSINT)} - verzamelen en analyseren van gegevens die verkregen
    zijn uit openbaar beschikbare bronnen
  \item \textbf{telefonie} - focust op de communicatie via mobiele apparaten
  \item \textbf{onderzoek gegevensdragers} - uitlezen van pc-apparatuur
  \item \textbf{hacking} - via hacking methodes informatie trachten te achterhalen
\end{itemize}

Sinds vijf jaar ziin er ook lokale CCU's, deze eenheid biedt ondersteuning aan de operationele
eenheden van de Lokale Politie. Francis werkt als hoofdinspecteur bij de LCCU van de politiezone
Grensleie. Daar is hij verantwoordelijk voor de bijzondere operaties en middelen (BOM), wat in de
praktijk overeenkomt met technische bijstand aan operaties.

Bij de oprichting van de CCU was er nog geen technologie zoals we die vandaag de dag kennen. Een van
hun hoofdtaken was toen het kopieren van harde schijven. De reden waarom ze de schijf kopieren, is
dat sommige schijven niet makkelijk te ontkoppelen zijn zonder eventuele schade aan te brengen. Als
er schade is, dan moeten zij ook de kosten dekken.

Tijdens het onderzoek gebruiken ze vaak de Linux distributie genaamd CAINE Forensics. De distributie
komt met een heleboel professionele tools speciaal gemaakt voor forensisch onderzoek uit te voeren.
DD, DCFLDD, DDRESCUE zijn voorbeelden van command-line tools die mee met de distributie komen. Die
tools gebruiken ze voor het kopieren van data. Het voordeel van zo'n tools te gebruiken, is dat ze
ruwe data kunnen kopieren zonder rekening te houden met het bestandssysteem. Een ander voordeel is
dat het de mogelijkheid biedt voor volledige schijven te klonen of weg te schrijven naar een
bestand. Bij het klonen schijven ze de kopie weg naar een EWF formaat die compressie mogelijk maakt.
Compressie is belangrijk, het bespaart hun wekelijks honderden gigabytes.

Sleuthkit is een andere tool die wordt meegeleverd bij CAINE. Het bevat een reeks command-line tools
die gewiste bestanden kunnen oplijsten en terughalen. Photorec is een van die tools, kan gebruikt
worden om bestanden op header niveau te gaan zoeken. Die techniek wordt ook wel carving genoemd.

Helaas zijn er ook situaties waar de tools niet kunnen gebruikt worden. Zo heeft Apple een
versleuteld besturingssyteem genaamd APFS. Hier kunnen ze niks mee aanvangen, tenzij ze het
wachtwoord weten. Ook bij eigen gemaakte bestandssytemen zoals in bepaalde GPS systemen en
bewakingscamera's werken de tools niet.

Naast het kopieren van data, onderzoeken ze ook telefoons. De twee commerciele produkten die ze
hiervoor gebruiken zijn Ufed en XRY. Het nadeel van commerciele produkten is dat de licencies niet
goedkoop zijn. Er zijn ook gratis manieren om data te kopieren van mobiele toestellen zoals
bijvoorbeeld Android Backup Toolkit. Ze hebben hiervoor wel de toegangscode nodig voor het toestel.
Daarom gebruiken ze sinds kort GrayKey, een tool die voor hun op een onbekende manier het toestel
kan ontgrendelen.

Na de uitleg over CAINE forensics werd er de wettelijke mogelijkheden en beperkingen toegelicht. De
grootste beperking momenteel is het aanvragen van de logs over een IP adres. Sinds kort is er een
dataretentie wetgeving die zegt dat providers maar maximaal zes maanden de data van een IP adres
mogen bijhouden. Zes maanden is vaak veel te kort, al deze opvragingen dienen ook te gebeuren via
het ministerie van buitenlandse zaken en dit kan soms maanden duren. Ze mogen dit ook niet altijd
opvragen, hiervoor hebben ze een vordering nodig van de procureur of de onderzoeksrechter. Een
procureur kan identificatie doen van een IP adres, een onderzoeksrechter daarintegen kan enkel
logbestanden opvragen van een IP adres. 

Francis vertelde ook dat een procureur een systeembeheerder kan vorderen. Het gaat hier meestal om
toegang verlenen tot bepaalde servers van het bedrijf of databanken. Indien de systeembeheerder geen
toegang verleent kan hij gestraft worden met een gevangenisstraf van zes tot drie jaar en met een
geldboete van zesentwintig euro tot twintigdeuizend euro of met een van die straffen alleen.

Vervolgens werd er kort procedures toegelicht met betrekking tot telefonie. Zo kan alleen een
officier van de gerechtelijke politie (OGP) kan een toestel in beslag nemen. Het toestel moet dan
ook direct in een toestand gebracht worden zodat er geen extern contact meer mogelijk is. Als er
contact zou gemaakt zijn, is er kans dat er gegevens gesynchroniseerd kunnen worden met de cloud en
zorgt ervoor dat de inbeslagname ongeldig is.

Op het einde van de sessie besprak hij de huidige fenomenen. Hacking is grotendeels verleden tijd,
de meeste systemen/websites zijn tegenwoordig zeer goed beveiligd. De huidige fenomenen zijn eerder
ransomware en oplichtingen.
